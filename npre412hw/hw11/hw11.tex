%        File: hw11.tex
%     Created: Sat Sep 10 10:00 AM 2016 C
% Last Change: Sat Sep 10 10:00 AM 2016 C
%

% use the answers clause to get answers to print; otherwise leave it out.
\documentclass[11pt,answers,addpoints]{exam}
% \documentclass[11pt,addpoints]{exam}
\RequirePackage{amssymb, amsfonts, amsmath, latexsym, verbatim, xspace, setspace}

% By default LaTeX uses large margins.  This doesn't work well on exams; problems
% end up in the "middle" of the page, reducing the amount of space for students
% to work on them.
\usepackage[margin=1in]{geometry}
\usepackage{enumerate}
\usepackage{minted}
\usepackage{hyperref}

% Here's where you edit the Class, Exam, Date, etc.
\newcommand{\class}{NPRE412}
\newcommand{\term}{Spring 2025}
\newcommand{\assignment}{HW 11}
\newcommand{\duedate}{2025.04.22}
%\newcommand{\timelimit}{50 Minutes}

\newcommand{\nth}{n\ensuremath{^{\text{th}}} }
\newcommand{\ve}[1]{\ensuremath{\mathbf{#1}}}
\newcommand{\Macro}{\ensuremath{\Sigma}}
\newcommand{\vOmega}{\ensuremath{\hat{\Omega}}}

% For an exam, single spacing is most appropriate
\singlespacing
% \onehalfspacing
% \doublespacing

% For an exam, we generally want to turn off paragraph indentation
\parindent 0ex

%\unframedsolutions
\usepackage{bibentry}

\usepackage{graphicx}
\begin{document} 


% These commands set up the running header on the top of the exam pages
\pagestyle{head}
\firstpageheader{}{}{}
\runningheader{\class}{\assignment\ - Page \thepage\ of \numpages}{Due \duedate}
\runningheadrule

\class \hfill \term \\
\assignment \hfill Due \duedate\\
\rule[1ex]{\textwidth}{.1pt}
%\hrulefill

%%%%%%%%%%%%%%%%%%%%%%%%%%%%%%%%%%%%%%%%%%%%%%%%%%%%%%%%%%%%%%%%%%%%%%%%%%%%%%%%%%%%%
%%%%%%%%%%%%%%%%%%%%%%%%%%%%%%%%%%%%%%%%%%%%%%%%%%%%%%%%%%%%%%%%%%%%%%%%%%%%%%%%%%%%%
\begin{itemize}
        \item Show your work.
        \item This work must be submitted online as a \texttt{.pdf} through
                Canvas.
        \item Work completed with LaTeX or Jupyter earns 1 extra point. Submit
                source file (e.g. \texttt{.tex} or \texttt{.ipynb}) along with
                the \texttt{.pdf} file.
        \item If this work is completed with the aid of a numerical program
                (such as Python, Wolfram Alpha, or MATLAB) all scripts and data
                must be submitted in addition to the \texttt{.pdf}.
        \item If you work with anyone else, document what you worked on together.
\end{itemize}
\rule[1ex]{\textwidth}{.1pt}



% ---------------------------------------------
\begin{questions}
        % ---------------------------------------------
	\question[15] Which is more dangerous, 1g of $^{99}Tc$ or 1mg of
	$^{137}Cs$? The Derived Air Concentrations (DAC) for these two isotopes
        are $3\times 10^{-7} \left[\frac{\mu Ci}{cm^3}\right]$ and 
        $6\times10^{-8} \left[\frac{\mu Ci}{cm^3}\right]$
        respectively.

        \begin{solution}
            With these values, Cs more toxic and ~25 times more toxic than Tc. 
            \begin{itemize}
                \item Toxicity of Tc: 5.68e4 m$^3$
                \item Toxicity of Cs: 1.44e6 m$^3$
            \end{itemize}

            Google the specific activities in $\mu Ci / g$
            \begin{itemize}
                \item $\mu$Ci-to-Bq = 37e3
                \item SA$_{Tc}$ = 630e6 / \ $\mu$Ci-to-Bq  [uCi/g]
                \item SA$_{Cs}$ = 3.2e12 / \ $\mu$Ci-to-Bq  [uCi/g]
            \end{itemize}

            Using Eq. 9.1, multiply the given masses by the specific activities and divide by the DAC in $\mu Ci/cc$.
            \begin{equation}
                Toxicity_i = \frac{A_i}{DAC_i} \frac{1 m^3}{1e6 \ cc}
            \end{equation}      
        \end{solution}

        % ---------------------------------------------
        \question Characterize the materials listed below as LLW, TRU, or HLW 
        (Tsoulfanidis, Question 9.3):


        \begin{parts}
        \part[5] gloves contaminated with $^{60}Co$ and $10 Ci$ of fission products 
        \begin{solution}
                LLW
        \end{solution}
        \part[5] a fuel rod from a BWR after 100 MWd/t burnup
        \begin{solution}
                HLW
        \end{solution}
        \part[5] shoe covers sprayed with tritiated ($^3H_2O$)water
        \begin{solution}
                LLW
        \end{solution}
        \part[5] uranium mill tailings
        \begin{solution}
                trick question, millings have their own classification. If I had to choose one of these, I would choose TRU.
        \end{solution}
        \part[5] 5g of irradiated LWR fuel containing 550 nCi of $^{252}$Cf.
        \begin{solution}
                TRU
        \end{solution}
        \end{parts}

        % ---------------------------------------------
        \question Describe (in less than 10 words each) the main drawback of 
        each of the following alternative spent nuclear fuel disposal 
        locations:
        \begin{parts}
                \part[5] Space.
        \begin{solution}
                Space-crafts costs too much
        \end{solution}
                \part[5] Deep seabed.
        \begin{solution}
                Outlawed by Congress (MPRSA), not enough research
        \end{solution}
                \part[5] Polar Ice Sheets.
        \begin{solution}
                Limited retrivability, uncertainty in ice-cap movement
        \end{solution}
                \part[5] Surface of a remote island.
        \begin{solution}
                Unpredictable tropical weather, lack of infastructure, transportation costs, pirates
        \end{solution}
        \end{parts}

        % ---------------------------------------------
        \question 
        In any generic mined geologic repository design, many engineered barriers defend 
        against the release of spent nuclear fuel into the geologic host media. 
        \begin{parts}
                \part[5] Name three geologic host media that we discussed in class.
        \begin{solution}
                salt, tuff, granite
        \end{solution}
                \part[5] List as many layers of engineered barriers as you can.
        \begin{solution}
                drift wall, drip shield, glass/ceramic waste package, gamma shield (lead), impact limiter, neutron shield (concrete), shear pads, shield plug, closure lid
        \end{solution}
                \part[10] Draw a diagram and label the placement of these 
                engineered and natural barriers together. 
        \begin{solution}
        
                \includegraphics[width=\linewidth]{IMG_5289.jpg}
        \end{solution}
        \end{parts}
        % ---------------------------------------------
        \question[15] A power reactor operated for 300 effective full-power days 
        (EFPD) at 1050 MW(e) with an efficiency of 33\%. What is the decay 
        power generated in the core 20 min after shutdown? Assume only 235U 
        fissions. (Tsoulfanidis, Question 9.6) (Hint: Use equation 9.9).
        \begin{solution}
                Power 20 min after shutdown is 54.13 MW$_{th}$. \\

                Use Eq. 9.7 for the function $F(t, \infty)$
                \begin{equation}
                    F(t, \infty) = \sum_{i=1}^{23} \frac{\alpha_i}{\lambda_i} e^{-\lambda_i t} 
                \end{equation}
                Use Eq. 9.9 for the power as a function of time.
                \begin{equation}
                    P(t, T) = \frac{P_0}{Q} \left[ F(t, \infty) - F(t + T, \infty) \right]
                \end{equation}
                Using T = 300 days = 2.592e7 s, t = 20 min = 1200 s, P0 = 1050 MW$_{e}$ / 0.33, Q~=~203~MeV.
        \end{solution}
        \question[5] We talked in class about the reference man. Who is the 
        reference man?
        \begin{solution}
                The reference man is the average man used in dosimitry and toxicity calculations. He is meant to be a good representation of the average person.
        \end{solution}

\end{questions}

\end{document}
