%        File: hw09.tex
%     Created: Sat Sep 10 10:00 AM 2016 C
% Last Change: Sat Sep 10 10:00 AM 2016 C
%

% use the answers clause to get answers to print; otherwise leave it out.
\documentclass[11pt,answers,addpoints]{exam}
%\documentclass[11pt,addpoints]{exam}
\RequirePackage{amssymb, amsfonts, amsmath, latexsym, verbatim, xspace, setspace}

% By default LaTeX uses large margins.  This doesn't work well on exams; problems
% end up in the "middle" of the page, reducing the amount of space for students
% to work on them.
\usepackage[margin=1in]{geometry}
\usepackage{enumerate}
\usepackage{minted}
\usepackage{hyperref}

% Here's where you edit the Class, Exam, Date, etc.
\newcommand{\class}{NPRE412}
\newcommand{\term}{Spring 2025}
\newcommand{\assignment}{HW 10}
\newcommand{\duedate}{2025.04.08}
%\newcommand{\timelimit}{50 Minutes}

\newcommand{\nth}{n\ensuremath{^{\text{th}}} }
\newcommand{\ve}[1]{\ensuremath{\mathbf{#1}}}
\newcommand{\Macro}{\ensuremath{\Sigma}}
\newcommand{\vOmega}{\ensuremath{\hat{\Omega}}}

% For an exam, single spacing is most appropriate
\singlespacing
% \onehalfspacing
% \doublespacing

% For an exam, we generally want to turn off paragraph indentation
\parindent 0ex

%\unframedsolutions
\usepackage{bibentry}
\begin{document} 


% These commands set up the running header on the top of the exam pages
\pagestyle{head}
\firstpageheader{}{}{}
\runningheader{\class}{\assignment\ - Page \thepage\ of \numpages}{Due \duedate}
\runningheadrule

\class \hfill \term \\
\assignment \hfill Due \duedate\\
\rule[1ex]{\textwidth}{.1pt}
%\hrulefill

%%%%%%%%%%%%%%%%%%%%%%%%%%%%%%%%%%%%%%%%%%%%%%%%%%%%%%%%%%%%%%%%%%%%%%%%%%%%%%%%%%%%%
%%%%%%%%%%%%%%%%%%%%%%%%%%%%%%%%%%%%%%%%%%%%%%%%%%%%%%%%%%%%%%%%%%%%%%%%%%%%%%%%%%%%%
\begin{itemize}
        \item Show your work.
        \item This work must be submitted online as a \texttt{.pdf} through
                Canvas.
        \item Work completed with LaTeX or Jupyter earns 1 extra point. Submit
                source file (e.g. \texttt{.tex} or \texttt{.ipynb}) along with
                the \texttt{.pdf} file.
        \item If this work is completed with the aid of a numerical program
                (such as Python, Wolfram Alpha, or MATLAB) all scripts and data
                must be submitted in addition to the \texttt{.pdf}.
        \item If you work with anyone else, document what you worked on together.
\end{itemize}
\rule[1ex]{\textwidth}{.1pt}



% ---------------------------------------------
\begin{questions}
        % ---------------------------------------------
        \question[50] What is the savings in natural uranium if both uranium 
        and plutonium are recycled in LWRs, assuming the following 
        (Tsoulfanidis, 7.3): 
        \begin{itemize}
                \item $3\%$ enriched fuel 
                \item with $0.22\%$ tails
                \item $0.78\%$ $^{235}U$ in spent fuel
                \item $6.9\frac{gfPu}{kg_{SNF}}$
                \item $0.90\frac{kg_{recovered}}{kg_{fresh}}$
                \item 0.8 Pu-$^{235}U$ equivalence.
        \end{itemize}
        \begin{solution}
        For the given specifications, the fuel savings is 36.0\%.

        
        To solve, first calculate the feed factor w/ Eq. 3.6:
        \begin{equation}
            FF = \frac{x_p - x_w}{x_f - x_w}
        \end{equation}
        Then use Eq. 7.12 to find the percent savings.
        \begin{equation}
            savings = \frac{u \cdot s \cdot p}{x_p - x_w} + \frac{u \left(x_s - x_w \right)}{x_p - x_w} = 36.0\%
        \end{equation}
        \end{solution}
        % ---------------------------------------------
        \question[50] What are the SWU savings for the for the conditions given 
        in the previous problem? (Tsoulfanidis, 7.4) 
        \begin{solution}
        By recycling used nuclear fuel, we can save 19.475\% of the SWU used to fabricate the fuel from natural uranium.

        To solve, use Eq 3.10 to find the separation potentials:
        \begin{equation}
            V(x) = (2 x - 1) \ln{\frac{x}{1 - x}}
        \end{equation}
        Then, find the SWU factor using Eq. 3.11:
        \begin{equation}
            SF = V(x_p) + \frac{W}{P} V(x_w) - \frac{f}{p} V(x_f)
        \end{equation}
        Finally, find the SWU saved using Eq. 7.16:
        \begin{equation}
            savings_{SWU} = \frac{u}{x_p - x_w} \left[
            s \cdot p + (x_s + x_w) \left( 1 - \frac{SF_s}{SF} \right)
            \right] = 19.475\%
        \end{equation}
        \end{solution}

\end{questions}

\end{document}
