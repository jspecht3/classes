%        File: hw13.tex
%     Created: Sat Sep 10 10:00 AM 2016 C
% Last Change: Sat Sep 10 10:00 AM 2016 C
%

% use the answers clause to get answers to print; otherwise leave it out.
\documentclass[11pt,answers,addpoints]{exam}
%\documentclass[11pt,addpoints]{exam}
\RequirePackage{amssymb, amsfonts, amsmath, latexsym, verbatim, xspace, setspace}

% By default LaTeX uses large margins.  This doesn't work well on exams; problems
% end up in the "middle" of the page, reducing the amount of space for students
% to work on them.
\usepackage[margin=1in]{geometry}
\usepackage{enumerate}
\usepackage{minted}
\usepackage{graphicx}
\usepackage{hyperref}


% Here's where you edit the Class, Exam, Date, etc.
\newcommand{\class}{NPRE412}
\newcommand{\term}{Spring 2025}
\newcommand{\assignment}{HW 13}
\newcommand{\duedate}{2025.05.29}
%\newcommand{\timelimit}{50 Minutes}

\newcommand{\nth}{n\ensuremath{^{\text{th}}} }
\newcommand{\ve}[1]{\ensuremath{\mathbf{#1}}}
\newcommand{\Macro}{\ensuremath{\Sigma}}
\newcommand{\vOmega}{\ensuremath{\hat{\Omega}}}

% For an exam, single spacing is most appropriate
\singlespacing
% \onehalfspacing
% \doublespacing

% For an exam, we generally want to turn off paragraph indentation
\parindent 0ex

%\unframedsolutions
\usepackage{bibentry}
\begin{document} 


% These commands set up the running header on the top of the exam pages
\pagestyle{head}
\firstpageheader{}{}{}
\runningheader{\class}{\assignment\ - Page \thepage\ of \numpages}{Due \duedate}
\runningheadrule

\class \hfill \term \\
\assignment \hfill Due \duedate\\
\rule[1ex]{\textwidth}{.1pt}
%\hrulefill

%%%%%%%%%%%%%%%%%%%%%%%%%%%%%%%%%%%%%%%%%%%%%%%%%%%%%%%%%%%%%%%%%%%%%%%%%%%%%%%%%%%%%
%%%%%%%%%%%%%%%%%%%%%%%%%%%%%%%%%%%%%%%%%%%%%%%%%%%%%%%%%%%%%%%%%%%%%%%%%%%%%%%%%%%%%
\begin{itemize}
        \item Show your work.
        \item This work must be submitted online as a \texttt{.pdf} through
                Canvas.
        \item Work completed with LaTeX or Jupyter earns 1 extra point. Submit
                source file (e.g. \texttt{.tex} or \texttt{.ipynb}) along with
                the \texttt{.pdf} file.
        \item If this work is completed with the aid of a numerical program
                (such as Python, Wolfram Alpha, or MATLAB) all scripts and data
                must be submitted in addition to the \texttt{.pdf}.
        \item If you work with anyone else, document what you worked on together.
\end{itemize}
\rule[1ex]{\textwidth}{.1pt}


% ---------------------------------------------
\begin{questions}
        % ---------------------------------------------
        \question In class, we discussed Megatons to Megawatts. The program 
        claims that 500 tons of HEU (90\% enriched) was converted to 7 billion 
        MWh.  
        \begin{parts}
                \part[10] Under the following assumptions, how many MWh would 
                have been generated? (\textbf{you must show your work.})
                \begin{itemize}
                        \item depleted uranium for downblending was 0.2\% enriched
			\item enrichment of LEU sent to U.S. 5\%
			\item avg. U.S. capacity factor $90\%$
                        \item avg. U.S. thermal efficiency $33\%$
                        \item avg. U.S. burnup 50,000 MWd/MTIHM
		\end{itemize}

        \begin{solution}
        Mahmoud and I got 3.70e9 MWh.

        Use feed factor equation.
        \begin{equation}
            ff = \frac{x_p - x_w}{x_f - x_w}
        \end{equation}
        Divide mass (m) by feed factor to get the mass of fuel.
        \begin{equation}
            M = \frac{m}{ff}
        \end{equation}
        Find power.
        \begin{equation}
            P = M \cdot \eta \cdot BU \cdot 24\frac{hours}{day} = 3.7e9 \ MWh
        \end{equation}
        \end{solution}

\part[20] Make some different \textbf{but still reasonable} assumptions that 
will result in the claim made by the program (500 tons of 90\% enriched HEU 
ultimately produced 7billion MWh). \textbf{List all assumptions} and 
\textbf{show} that the resulting calculation now gives approximately 7billion 
MWh generated. 
        \begin{solution}
        To get the values the government got, assume:
        \begin{itemize}
            \item the product enrichment is 3.5\%, the lower bound of LWR enrichment,
            \item $\eta$ is 35.75\% as Rankine turbines have gotten better at extracting heat, and
            \item BU is 60 because we got better at loading the cores.
        \end{itemize}

        Next, find the new feed factor
        \begin{equation}
            ff = \frac{x_p - x_w}{x_f - x_w} = 0.0367
        \end{equation}
        Find the new mass of the fuel.
        \begin{equation}
            M = \frac{m}{ff} = 1.36e7 \ kg
        \end{equation}
        Find new power
        \begin{equation}
            P = M \cdot \eta_2 \cdot BU_2 \cdot 24 \frac{hours}{day} = 7e9 MWh
        \end{equation}
        \end{solution}
        \end{parts}
        % ---------------------------------------------
        \question[10] How many significant quantities are in 500 tons of 90\% 
        enriched HEU?

        \begin{solution}
        Mahmoud and I got 18,000 significant quantities.

        We know the significant quantity of direct use U-235 is 25 kg. Use this to find the number of significant quantities.
        \begin{equation}
            SQs = \frac{m \cdot x_f}{m_{sq}} = \frac{500e3 \ kg \cdot 0.9}{25 \ kg}
        \end{equation}
        \end{solution}

        % ---------------------------------------------
        \question[10] How many significant quantities are in 500 tons of natural 
        thorium?

        \begin{solution}
        Mahmoud and I got 25 significant quantities.

        We know the significant quantity of direct use U-235 is 25 kg. Use this to find the number of significant quantities.
        \begin{equation}
            SQs = \frac{m}{m_sq} = \frac{500e3 \ kg}{20e3 \ kg}
        \end{equation}
        \end{solution}

       % ---------------------------------------------
        \question You are an inspector at Natanz.

        On February 1: 
        \begin{itemize}
        \item You concluded that there were 4 canisters on site containing 5\% 
                enriched uranium.
        \item They collectively hold 8.2kg of $^{235}U$.
        \item You tag the canisters with tamper-resistant seals.
        \end{itemize}

        On March 1:
        \begin{itemize}
        \item You count 4 canisters. 
        \item Three of the original 4 canisters are present and remain tagged. 
        \item The missing one contained 1kg of $^{235}U$ (according to your records).
        \item You are told that in February,  enriched uranium was created, 
                resulting in 2kg $^{235}U$, placed in the new, untagged canister.
        \item And, you are shown that the missing 1kg $^{235}U$ canister had been shipped to a reactor facility. 
        \end{itemize}
        \begin{parts}
                \part[10] What is the book inventory?
        \begin{solution}
        The book inventory is 9.2 kg.

        \begin{equation}
            BI = PB + X - Y = 8.2 + 2 - 1 = 9.2 \ kg
        \end{equation}
        \end{solution}

                \part[10] You take 10 readings of the mass of $^{235}U$ in 
                the new canister. They give: 2.102, 2.015, 2.022, 1.998, 1.989, 
                2.101, 2.077, 1.970, 1.92, 2.01. What is the mean of the MUF? 
        \begin{solution}
        The mean MUF is 0.0204 kg.

        Find the means using numpy.
        \begin{equation}
            means = np.means(masses)
        \end{equation}
        Find the MUF.
        \begin{equation}
            MUF = means - 2 = 0.0204 = kg.
        \end{equation}
        \end{solution}

                \part[10] What is $\sigma_{MUF}$?
        \begin{solution}
        The standard deviation is 0.055 kg. I just used numpy to find the standard deviation.

        \end{solution}
                \part[10] Has this inspection broken the three-sigma rule?
        \begin{solution}
        No. MUF is less than 3 $\sigma$. 

        \begin{equation}
        0.0204 \ kg < 0.166 \ kg
        \end{equation}
        \end{solution}

                \part[10] If all following months have a MUF exactly like this, 
                how many months until $MUF_c \ge 1 SQ$?
        \begin{solution}
        It would take 73,529.4 months to get a significant quantity.

        For indirect use, enrichment below 20\%, the significant quantity is 75 kg. Use this to find the months required.
        \begin{equation}
            M = \frac{SQ_{U235}}{5e-2 \cdot MUF}
        \end{equation}
        \end{solution}

        \end{parts}


\end{questions}

\end{document}
