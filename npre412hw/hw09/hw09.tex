%        File: hw09.tex
%     Created: Sat Sep 10 10:00 AM 2016 C
% Last Change: Sat Sep 10 10:00 AM 2016 C
%

% use the answers clause to get answers to print; otherwise leave it out.
\documentclass[11pt,answers,addpoints]{exam}
%\documentclass[11pt,addpoints]{exam}
\RequirePackage{amssymb, amsfonts, amsmath, latexsym, verbatim, xspace, setspace}

% By default LaTeX uses large margins.  This doesn't work well on exams; problems
% end up in the "middle" of the page, reducing the amount of space for students
% to work on them.
\usepackage[margin=1in]{geometry}
\usepackage{enumerate}
\usepackage{minted}
\usepackage{hyperref}

% Here's where you edit the Class, Exam, Date, etc.
\newcommand{\class}{NPRE412}
\newcommand{\term}{Spring 2025}
\newcommand{\assignment}{HW 9}
\newcommand{\duedate}{2025.04.01}
%\newcommand{\timelimit}{50 Minutes}

\newcommand{\nth}{n\ensuremath{^{\text{th}}} }
\newcommand{\ve}[1]{\ensuremath{\mathbf{#1}}}
\newcommand{\Macro}{\ensuremath{\Sigma}}
\newcommand{\vOmega}{\ensuremath{\hat{\Omega}}}

% For an exam, single spacing is most appropriate
\singlespacing
% \onehalfspacing
% \doublespacing

% For an exam, we generally want to turn off paragraph indentation
\parindent 0ex

%\unframedsolutions
\usepackage{bibentry}
\begin{document} 


% These commands set up the running header on the top of the exam pages
\pagestyle{head}
\firstpageheader{}{}{}
\runningheader{\class}{\assignment\ - Page \thepage\ of \numpages}{Due \duedate}
\runningheadrule

\class \hfill \term \\
\assignment \hfill Due \duedate\\
\rule[1ex]{\textwidth}{.1pt}
%\hrulefill

%%%%%%%%%%%%%%%%%%%%%%%%%%%%%%%%%%%%%%%%%%%%%%%%%%%%%%%%%%%%%%%%%%%%%%%%%%%%%%%%%%%%%
%%%%%%%%%%%%%%%%%%%%%%%%%%%%%%%%%%%%%%%%%%%%%%%%%%%%%%%%%%%%%%%%%%%%%%%%%%%%%%%%%%%%%
\begin{itemize}
        \item Show your work.
        \item This work must be submitted online as a \texttt{.pdf} through
                Canvas.
        \item Work completed with LaTeX or Jupyter earns 1 extra point. Submit
                source file (e.g. \texttt{.tex} or \texttt{.ipynb}) along with
                the \texttt{.pdf} file.
        \item If this work is completed with the aid of a numerical program
                (such as Python, Wolfram Alpha, or MATLAB) all scripts and data
                must be submitted in addition to the \texttt{.pdf}.
        \item If you work with anyone else, document what you worked on together.
\end{itemize}
\rule[1ex]{\textwidth}{.1pt}



% ---------------------------------------------
\begin{questions}
        % ---------------------------------------------
        \question[50] What is the value of uranium utilization in LWRs, for a 
        once-through fuel cycle, if the fuel acheives a burnup of 45000 MWd/tU? 
        Assume enrichment of fuel is $3.2\%$ and tails are equal to $0.28\%$ 
        (Tsoulfanidis, 7.1).
        \begin{solution}
        The uranium utilization for LWRs is 0.815\% for the given specifications.

        To solve, first calculate the feed factor:
        \begin{equation}
            FF = \frac{x_p - x_w}{x_f - x_w}
        \end{equation}
        Then, calculate the mass ratio of uranium after the cycle finishes to when you load the cycle as a function of burnup.
        \begin{equation}
            m_r = BU \cdot 1.23 \frac{g}{MWd} \cdot 1e-6 \frac{ton}{gram} = \frac{M_f}{M_l}
        \end{equation}
        Finally, calculate the utilization using Eq. 7.2:
        \begin{equation}
            u_{lwr} = \frac{m_r}{FF} = 0.815\%
        \end{equation}
        \end{solution}
        % ---------------------------------------------
        \question[50] Repeat the previous problem for a breeder reactor. Assume 
        $\gamma = 0.02$ (Tsoulfanidis, 7.2).
        \begin{solution}
        For a breeder reactor with the given specifications, the utilization is 74.55\%.

        Using Eq. 7.3:
        \begin{equation}
            u_{breeder} = \frac{m_r}{m_r (1 - \gamma) + \gamma} = 74.55\%
        \end{equation}
        \end{solution}

\end{questions}

\end{document}
